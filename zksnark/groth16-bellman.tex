\documentclass{article}
\usepackage[UTF8, heading = false, scheme = plain]{ctex}

\usepackage{geometry}
\geometry{b5paper,left=2cm,right=2cm,top=2cm,bottom=2cm}

\usepackage{color}
\usepackage{amsfonts}
\usepackage{amsmath}

\linespread{1.5}

\usepackage[colorlinks,
            linkcolor=red,
            anchorcolor=blue,
            citecolor=green
            ]{hyperref}

\usepackage{listings}
\usepackage{fontspec}
\usepackage{graphicx}
\usepackage{algorithmic}
\newfontfamily\monaco{Monaco}
\definecolor{dkgreen}{rgb}{0,0.6,0}
\definecolor{gray}{rgb}{0.5,0.5,0.5}
\definecolor{mauve}{rgb}{0.58,0,0.82}
\lstset{ %
  basicstyle=\footnotesize\monaco,       % the size of the fonts that are used for the code
  numbers=left,                   % where to put the line-numbers
  numberstyle=\footnotesize\monaco\color{gray},  % the style that is used for the line-numbers
  numbersep=5pt
  stepnumber=1,                   % the step between two line-numbers. If it's 1, each line
                                  % will be numbered
  numbersep=5pt,                  % how far the line-numbers are from the code
  backgroundcolor=\color{white},      % choose the background color. You must add \usepackage{color}
  showspaces=false,               % show spaces adding particular underscores
  showstringspaces=false,         % underline spaces within strings
  showtabs=false,                 % show tabs within strings adding particular underscores
  frame=lines,                   % adds a frame around the code
  rulecolor=\color{black},        % if not set, the frame-color may be changed on line-breaks within not-black text (e.g. commens (green here))
  tabsize=4,                      % sets default tabsize to 2 spaces
  captionpos=t,                   % sets the caption-position to bottom
  breaklines=true,                % sets automatic line breaking
  breakatwhitespace=false,        % sets if automatic breaks should only happen at whitespace
  title=\lstname,                   % show the filename of files included with \lstinputlisting;
                                  % also try caption instead of title
  keywordstyle=\color{blue},          % keyword style
  commentstyle=\color{dkgreen},       % comment style
  stringstyle=\color{mauve},         % string literal style
  escapeinside={\%*}{*)},            % if you want to add LaTeX within your code
  morekeywords={*,...}               % if you want to add more keywords to the set
}

\usepackage{amssymb} 
\usepackage{amsmath}
\usepackage[ruled,vlined]{algorithm2e}

\setlength{\parindent}{2em}

\renewcommand{\G}{\mathbb{G}}
\newcommand{\Z}{\mathbb{Z}}
\newcommand{\Zn}{\mathbb{Z}_n}
\newcommand{\Q}{\mathbb{Q}}

\newcommand{\F}{\mathbb{F}}
\newcommand{\Fq}{\mathbb{F}_q}
\newcommand{\Fp}{\mathbb{F}_p}

\newcommand{\Sbox}{\textsf{Sbox}}
\newcommand{\code}[1]{\lstinline!#1!}

\newcommand{\CKDpriv}{\textsf{CKDpriv}}
\newcommand{\CKDpub}{\textsf{CKDpub}}

%%%%%%%处理下划线:_%%%%%%%%%
\usepackage{underscore}
%%%%%%%处理下划线:_%%%%%%%%%

\setlength{\parindent}{2.1em}

%%%设置页眉和页码格式
\usepackage{fancyhdr}
\newcommand{\makeheadrule}{%
\rule[0.85\baselineskip]{\headwidth}{0.5pt}\vskip-.8\baselineskip}%1.5 0.4->0.5
\makeatletter
\renewcommand{\headrule}{%
{\if@fancyplain\let\headrulewidth\plainheadrulewidth\fi
\makeheadrule}}
\makeatother
\pagestyle{fancy}
\fancyhf{}
\fancyhead[r]{\textit{Crypto In Action}}
\fancyfoot[C]{--{~\thepage~}--}
%%%设置页眉和页码格式结束

\usepackage{color}
\newcommand{\red}{\textcolor{red}}
\newcommand{\blue}{\textcolor{blue}}

\begin{document}
\title{Groth16从原理到实现}
\author{hwp, longcpp\\ \small{leo.hou@tron.network, longcpp9@gmail.com}}

\maketitle
\section{预备知识}
\subsection{FFT/iFFT}
$\omega$是$N$次单位根,N是2的$n$次幂.
$$ f(X) \triangleq \sum_{i=0}^{N-1}{a_i\cdot X^i} $$
$$ b_j = f(\omega^j)=\sum_{i=0}^{N-1}{a_i\cdot \omega^{ji}}$$
$(b_0,b_1,..,b_{N-1})$是$(a_0,a_1,...,a_{N-1})$的离散傅里叶变换,可以用FFT快速实现。\\
$$
\begin{bmatrix}
\omega^{0\cdot0} & \omega^{0\cdot1} & ... & \omega^{0\cdot{N-1}}\\
\omega^{1\cdot0} & \omega^{1\cdot1} & ... & \omega^{1\cdot{N-1}}\\
\vdots & \\
\omega^{(N-1)\cdot0} & \omega^{(N-1)\cdot1} & ... & \omega^{(N-1)\cdot{N-1}}\\
\end{bmatrix}
\cdot 
\begin{pmatrix} a_0 \\ a_1 \\ \vdots \\a_{N-1}\\ \end{pmatrix} = 
\begin{pmatrix} b_0 \\ b_1 \\ \vdots \\b_{N-1}\\ \end{pmatrix}
$$
由于,
$$
\begin{bmatrix}
\omega^{-0\cdot0} & \omega^{-0\cdot1} & ... & \omega^{-0\cdot{N-1}}\\
\omega^{-1\cdot0} & \omega^{-1\cdot1} & ... & \omega^{-1\cdot{N-1}}\\
\vdots & \\
\omega^{-(N-1)\cdot0} & \omega^{-(N-1)\cdot1} & ... & \omega^{-(N-1)\cdot{N-1}}\\
\end{bmatrix}
\cdot
\begin{bmatrix}
\omega^{0\cdot0} & \omega^{0\cdot1} & ... & \omega^{0\cdot{N-1}}\\
\omega^{1\cdot0} & \omega^{1\cdot1} & ... & \omega^{1\cdot{N-1}}\\
\vdots & \\
\omega^{(N-1)\cdot0} & \omega^{(N-1)\cdot1} & ... & \omega^{(N-1)\cdot{N-1}}\\
\end{bmatrix}
=N\cdot I_{N}
$$
$$
\sum_{k=0}^{N-1}{\omega^{-ik}\cdot \omega^{kj}}=\sum_{k=0}^{N-1}{\omega^{k(j-i)}}
=
\begin{cases}
N &\text{if } i = j \\
0 &\text{if } i \not = j
\end{cases}
$$
所以,
$$
\begin{pmatrix} a_0 \\ a_1 \\ \vdots \\a_{N-1}\\ \end{pmatrix} 
= \frac{1}{N}
\begin{bmatrix}
\omega^{-0\cdot0} & \omega^{-0\cdot1} & ... & \omega^{-0\cdot{N-1}}\\
\omega^{-1\cdot0} & \omega^{-1\cdot1} & ... & \omega^{-1\cdot{N-1}}\\
\vdots & \\
\omega^{-(N-1)\cdot0} & \omega^{-(N-1)\cdot1} & ... & \omega^{-(N-1)\cdot{N-1}}\\
\end{bmatrix}
\cdot
\begin{pmatrix} b_0 \\ b_1 \\ \vdots \\b_{N-1}\\ \end{pmatrix}
$$
$(a_0,a_1,...,a_{N-1})$是$(b_0,b_1,..,b_{N-1})$的离散傅里叶变换的逆变换,可以用iFFT快速实现。
$$g(X) = \sum_{j=0}^{N-1}{b_j\cdot X^j}$$
$$a_i= \frac{1}{N}g(\omega^{-i})=\sum_{j=0}^{N-1}{b_j\cdot \omega^{-ij}}$$

\section{zk-snark}
Prover向Verifier证明他知道$C_1,C_2,C_3$使得$(C_1\cdot C_2)\cdot(C_1+C_3)=7$
\subsection{flatten}
$$S1 = C_1 \cdot C_2$$
$$S2 = C_1 + C_3$$
$$out = S1 \cdot S2$$

\subsection{R1CS}
定义$\overrightarrow{s}=[1,out,C_1,C_2,C_3,S_1,S_2]$
每个电路定义一组向量$\overrightarrow{u},\overrightarrow{v},\overrightarrow{w}$,
满足$\overrightarrow{s}\cdot\overrightarrow{u} * \overrightarrow{s}\cdot\overrightarrow{v} = \overrightarrow{s}\cdot\overrightarrow{w}$\\
第1个电路
$$\overrightarrow{u} = [0,0,1,0,0,0,0]$$
$$\overrightarrow{v} = [0,0,0,1,0,0,0]$$
$$\overrightarrow{w} = [0,0,0,0,0,1,0]$$
第2个电路
$$\overrightarrow{u} = [0,0,1,0,1,0,0]$$
$$\overrightarrow{v} = [1,0,0,0,0,0,0]$$
$$\overrightarrow{w} = [0,0,0,0,0,0,1]$$
第3个电路
$$\overrightarrow{u} = [0,0,0,0,0,1,0]$$
$$\overrightarrow{v} = [0,0,0,0,0,0,1]$$
$$\overrightarrow{w} = [0,1,0,0,0,0,0]$$

$$
U=\begin{bmatrix}
0&0&1&0&0&0&0\\
0&0&1&0&1&0&0\\
0&0&0&0&0&1&0\\
\end{bmatrix}
$$

$$
V=\begin{bmatrix}
0&0&0&1&0&0&0\\
1&0&0&0&0&0&0\\
0&0&0&0&0&0&1\\
\end{bmatrix}
$$

$$
W=\begin{bmatrix}
0&0&0&0&0&1&0\\
0&0&0&0&0&0&1\\
0&1&0&0&0&0&0\\
\end{bmatrix}
$$

\subsection{QAP}
定义
$$U(X)=[u_0(X),...,u_6(X)]$$
$$V(X)=[v_0(X),...,v_6(X)]$$
$$W(X)=[w_0(X),...,w_6(X)]$$
使得$X=1,2,3$时<$U(X),V(X),W(X)$>分别对应第1,2,3个电路的<$\overrightarrow{u},\overrightarrow{v},\overrightarrow{w}$ >
$u_i(X),v_i(X),w_i(X)$都可以用拉格朗日插值公式求。
为了快速计算多项式$u_i(X),v_i(X),w_i(X)$,可以选择$X=\omega^0,\omega^1,\omega^2$. 其中$\omega$为4次单位根,
即$\omega^4=1$

设$\overrightarrow{U_i}=[y_0,y_1,...,y_{N-1}]$,$N$为限制条件的个数,如果$N$不是2的次幂,则将$\overrightarrow{U_i}$用0补全。
根据$\overrightarrow{U_i}$计算$U_i(X)$,使得$U_i(\omega^j)=y_j$, $\omega$是$N$次单位根。

首先定义拉格朗日基${L_0(X),L_1(X),...,L_{N-1}(X)}$,使得
$$
L_i(\omega^j) = \begin{cases}
    0 &\text{if } i\not = j \\
    1 &\text{if } i=j
 \end{cases}
 $$
则  
$$U_i(X) = \sum_{j=0}^{N-1}{y_i\cdot L_j(X)}$$
$U_i(X)$在x点的值为: 
$$U_i(x)=\sum_{j=0}^{N-1}{y_i\cdot L_j(x)}.$$

下面介绍如何求$L_j(x)$.
$$L_j(X) = \frac{1}{N}{\sum_{i=0}^{N-1}({\frac{X}{\omega^j}})^i}$$
 定义
 $$P(Y)=\sum_{i=0}^{N-1}{(x\cdot Y)^i}=\sum_{i=0}^{N-1}{{x}^i\cdot Y^i} $$
则有
$$ L_j(x) =\frac{1}{N}\sum_{i=0}^{N-1}{(\frac{x}{\omega^j})^i}
=\frac{1}{N}\sum_{i=0}^{N-1}{{x}^i\cdot (\omega^{-j})^i}
=\frac{1}{N}{P(\omega^{-j})}$$
对$P(Y)$的系数$[x^0,...,x^{N-1}]$利用iFFT求$[L_0(x),...,L_{N-1}(x)]$.
实际上,
$$
\sum_{j=0}^{N-1} L_j(x)\omega^{i\cdot j}
=\sum_{j=0}^{N-1} (\frac{1}{N}\sum_{k=0}^{N-1}x^k \omega^{-j\cdot k}) \omega^{i\cdot j}
%=\frac{1}{N}\sum_{j=0}^{N-1}\sum_{k=0}^{N-1} x^k\cdot \omega^{(i-k)j}
=\frac{1}{N}\sum_{k=0}^{N-1}x^k\sum_{j=0}^{N-1} \cdot \omega^{(i-k)j}
$$

\begin{displaymath}
\sum_{j=0}^{N-1} \cdot \omega^{(i-k)j} = \left\{ \begin{array}{ll}
    N & k=i \\
    0 & k \not = i
\end{array} \right.
\end{displaymath}
$\sum_{j=0}^{N-1} L_j(x)\omega^{i\cdot j} = x^k$,$[x^0,...,x^{N-1}]$是$[L_0(x),...,L_{N-1}(x)]$的离散傅里叶变换。


\subsection{Groth16}

\subsubsection{setup}

在setup阶段生成CRS(Common reference string),即PK(Proving Key) 和 VK(Verifying Key).
其中VK为:
$$VK_\alpha = [\alpha]_1,\ VK'_\beta=[\beta]_2,\ VK'_\gamma=[\gamma]_2,\ VK'_\delta=[\delta]_2 $$
$$VK_{L,i} = \left[\frac{\beta u_i(x)+\alpha v_i(x)+w_i(x)}{\gamma}\right]_1, for\ i = 0,...,l$$
而PK为:
$$PK_\alpha=[\alpha]_1,\ PK_\beta=[\beta]_1,\ PK'_\beta=[\beta]_2,\ PK_\delta=[\delta]_1,\ PK'_\delta=[\delta]_2 $$
$$PK_{H,i}=[x^i]_1,\ PK'_{H,i} =[x^i]_2,\ for\ i=0,...,n-1$$
$$PK_{K,i} = \left[\frac{\beta u_i(x)+\alpha v_i(x)+w_i(x)}{\delta}\right]_1,\ for\ i=l+1,..,m$$
$$PK_{T,i} = \left[\frac{x^it(x)}{\delta}\right]_1,\ for\ i=0,..,n-2$$

Sapling的实现与Groth16略有不同。
\paragraph{Powers of Tau\\}

生成与应用无关的CRS.如$[\alpha]_1, [\beta]_1, [\beta]_2,\{[x^i]_1\}_{i=0}^{n-1}$.

为了能计算$\{[\frac{x^it(x)}{\delta}]_1\}_{i=0}^{n-2}$,还需要$\{[x^i]_1\}_{i=n}^{2n-2}$.

为了能利用FFT快速计算$\beta u_i(x),\alpha v_i(x), w_i(x)$,
需要计算拉格朗日基
$${\{L_i([x]_1)\}}_{i=0}^{n-1},{\{L_i([x]_2)\}}_{i=0}^{n-1},$$
以及 
$${\{\alpha L_i([x]_1)\}}_{i=0}^{n-1},{\{\beta L_i([x]_1)\}}_{i=0}^{n-1}$$

所以第一阶段结束之后生成的参数如下:\\
$[\alpha]_1, [\beta]_1, [\beta]_2$\\
$g1\_coeffs: {\{L_i([x]_1)\}}_{i=0}^{n-1}$\\
$g2\_coeffs: {\{L_i([x]_2)\}}_{i=0}^{n-1}$\\
$g1\_alpha\_coeffs: {\{\alpha\cdot L_i([x]_1)\}}_{i=0}^{n-1}$\\
$g2\_\beta\_coeffs: {\{\beta\cdot L_i([x]_1)\}}_{i=0}^{n-1}$\\
$h: {\{[x^it(x)]_1\}}_{i=0}^{n-2}$

\textbf{Sapling MPC}
Sapling MPC阶段生成剩余的CRS,
$$
[\delta]_1,[\delta]_2, 
\left\{\left[\frac{\beta u_i(x)+\alpha v_i(x)+w_i(x)}{\delta}\right]_1\right\}_{i=l+1}^m,
$$
$$
\left\{\left[\frac{x^it(x)}{\delta}\right]_1\right\}_{i=0}^{n-2},
\left\{[u_i(x)]_1\right\}_{i=0}^m, \left\{[v_i(x)]_1\right\}_{i=0}^m,\left\{[v_i(x)]_2\right\}_{i=0}^m
$$

\subsubsection{Proving}
随机选择$r,s\in \mathbb{Z}_p$,计算$proof=([A]_1,[B]_2,[C]_1)$.其中,\\
$$
A=\alpha +\sum_{i=0}^m{a_iu_i(x)+r\delta}
$$
$$
B=\beta +\sum_{i=0}^m{a_iv_i(x)+s\delta}
$$
$$
C=\sum_{i=l+1}^{m}{a_i\cdot \frac{\beta u_i(x)+\alpha v_i(x)+w_i(x)}{\delta}} + \frac{h(x)t(x)}{\delta}
+ A\cdot s + B\cdot r - rs\delta
$$

需要注意的是,无法先计算$A$,再计算$[A]_1$.因为是$\alpha,x,\delta$未知的, 而$[\alpha]_1,[\delta]_1$是已知的,
所以计算$[A]_1$可以通过以下等式.
$$
[A]_1=P K_{\alpha}+\sum_{j=0}^{n-1} \sum_{i=0}^{m} a_{i} u_{i, j} PK_{H, j}+r P K_{\delta}, 
\text{其中} [u_{i,0},u_{i,1},..,u_{i,n-1}] \text{为} u_i(X) \text{的系数}.
$$
$$
[B]_2=P K_{\beta}^{\prime}+\sum_{j=0}^{n-1} \sum_{i=0}^{m} a_{i} v_{i, j} PK_{H, j}^{\prime}+s P K_{\delta}^{\prime},
\text{其中} [v_{i,0},v_{i,1},..,v_{i,n-1}] \text{为} v_i(X) \text{的系数}.
$$
$$
[B]_1=P K_{\beta}+\sum_{j=0}^{n-1} \sum_{i=0}^{m} a_{i} v_{i, j} PK_{H, j}+s PK_{\delta},
\text{其中} [v_{i,0},v_{i,1},..,v_{i,n-1}] \text{为} v_i(X) \text{的系数}.
$$
$$
[C]_1 =\sum_{i=l+1}^{m}a_{i} PK_{K,i}+\sum_{i=0}^{n-2}h_{i}PK_{T,i}+s[A]_1+r[B]_1-rsPK_{\delta},
\text{其中}[h_0,h_1,..h_{n-2}]\text{为}h(x)\text{的系数}.
$$
在Sapling的实现中,在Sapling MPC阶段直接生成了$\{[u_i(x)]_1\}_{i=0}^m$,$\{[v_i(x)]_1\}_{i=0}^m$和$\{[v_i(x)]_2\}_{i=0}^m$.
所以在计算$[A]_1,[B]_1,[B]_2$时直接采用下列方式.
$$
[A]_1=P K_{\alpha}+\sum_{i=0}^{m} a_{i}\cdot u_i([x]_1)+r P K_{\delta}, 
$$
$$
[B]_2=P K_{\beta}^{\prime}+\sum_{i=0}^{m} a_{i}\cdot v_i([x]_2)+s P K_{\delta}^{\prime},
$$
$$
[B]_1=P K_{\beta}+\sum_{i=0}^{m} a_{i}\cdot v_i([x]_1)+s PK_{\delta},
$$
在计算$[C]_1$时,最复杂的是计算$h(x)\text{的系数}h_i$.
记
\begin{displaymath}
    \begin{array}{ll}
        \tilde{U}(X) &= [a_0,...,a_m] [u_0(X),..,u_m(X)]^T \\
        \tilde{V}(X) &= [a_0,...,a_m] [v_0(X),..,v_m(X)]^T \\
        \tilde{W}(X) &= [a_0,...,a_m] [w_0(X),..,w_m(X)]^T
    \end{array}
    \end{displaymath}
根据QAP等式,
$$\tilde{U}(X) \cdot  \tilde{V}(X) - \tilde{W}(X) = h(X)t(X)$$
$$h(X) = \frac{\tilde{U}(X) \cdot  \tilde{V}(X) - \tilde{W}(X)}{t(X)}$$

如果采用iFFT求$h_i$,则需要知道$h(\omega^0),h(\omega^1),...$.现在知道$\tilde{U}(\omega^j),\tilde{V}(\omega^j),
\tilde{W}(\omega^j),j=0,1,2..$, 但是$t(\omega^j)=0,j=1,1,2...$,没法直接利用iFFT.

通常的办法是找一个数$\sigma$,使得$t(\sigma\omega^j),j=0,1,2...$不为0.
先计算$h(\sigma\omega^j),j=0,1,2...$, 再利用iFFT计算。
这相当于是已知$\tilde{h}(X)=\sum_{i=0}^{n-2} h_i\sigma^i X^i$在$w^0,w^1,...$的值,
做iFFT求得$[h_0\sigma^0,h_1\sigma^1,...]$,再消去$\sigma$可以得到$h(x)$系数$h_i$.

综上,计算$h(x)$系数$h_i$过程如下。

根据输入和电路,可以计算出$\tilde{U}(w^0),\tilde{U}(w^1),...$,通过iFFT可以计算出$\tilde{U}(X)$的系数;
取$\mathbb{F}_r$的生成元为$\sigma$.再利用一次FFT计算$\tilde{U}(\sigma\omega^j),j=0,1,2...$的值.
同理可以求得$\tilde{V}(\sigma\omega^j),\tilde{W}(\sigma\omega^j),j=0,1,2...$的值.
计算$t(w^j)=\sigma^n-1,j=0,1,2,...$.

计算$h(\sigma\omega^j),j=0,1,2...$, 再利用iFFT计算$[h_0\sigma^0,h_1\sigma^1,...]$,再消去$\sigma$可以得到$h(x)$系数$h_i$.
整个过程需要3次FFT和4次iFFT.



\subsubsection{Verifying}
要验证proof,需要借助pairing, 如下所示,$e$为pairing函数。
$$
e([A]_1,[B]_2) = e(VK_\alpha,VK_\beta^{\prime})\cdot e(VK_x, VK_\gamma^{\prime})\cdot 
e([C]_1,VK_\delta^{\prime})
$$
其中,
$$VK_x = \sum_{i=0}^l a_i\cdot VK_{L,i} = \sum_{i=0}^l a_i\cdot 
[\frac{\beta u_i(x)+\alpha v_i(x)+w_i(x)}{\gamma}]_1$$
证明上式成立实际上是证明下列等式成立.
$$
A\cdot B=\alpha\cdot\beta + \sum_{i=0}^l a_i\cdot\frac{\beta u_i(x) + \alpha v_i(x) + w_i(x)}{\gamma} \cdot \gamma +
C\cdot\delta
$$
上式等式化简可以得到QAP的等式
$$(\sum_{i=0}^m a_i\cdot u_i(x))\cdot(\sum_{i=0}^m a_i\cdot u_i(x))=\sum_{i=0}^m a_i\cdot w_i(x)$$
在Sapling的实现中,取$\gamma=1$,即$[\gamma]_2 = g_2$,$g_2$是群$\G_2$的生成元。\\
验证等式变为
$$
e([A]_1,[B]_2) = e(VK_\alpha,VK_\beta^{\prime})\cdot e(VK_x, g_2)\cdot 
e([C]_1,VK_\delta^{\prime})
$$
其中,
$$VK_x = \sum_{i=0}^l a_i\cdot VK_{L,i} = \sum_{i=0}^l a_i\cdot 
(\beta u_i(x)+\alpha v_i(x)+w_i(x))_1$$
进一步变为
$$
e([A]_1,[B]_2) \cdot e(VK_x, -g_2)\cdot e([C]_1,-VK_\delta^{\prime}) = e(VK_\alpha,VK_\beta^{\prime}) 
$$
在进行$init\_zksnark\_params$预计算$e(VK_\alpha,VK_\beta^{\prime}),-g_2,-VK_\delta^{\prime}$
\section{代码}
以$c = a + b$为例


(To be continued...)

\end{document}   