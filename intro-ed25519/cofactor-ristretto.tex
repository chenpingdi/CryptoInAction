\documentclass{article}
\usepackage[UTF8, heading = false, scheme = plain]{ctex}

\usepackage{geometry}
\geometry{b5paper,left=2cm,right=2cm,top=2cm,bottom=2cm}

\usepackage{color}
\usepackage{amsfonts}
\usepackage{amsmath}

\linespread{1.5}

\usepackage[colorlinks,
            linkcolor=red,
            anchorcolor=blue,
            citecolor=green
            ]{hyperref}

\usepackage{listings}
\usepackage{fontspec}
\usepackage{graphicx}
\usepackage{algorithmic}
\newfontfamily\monaco{Monaco}
\definecolor{dkgreen}{rgb}{0,0.6,0}
\definecolor{gray}{rgb}{0.5,0.5,0.5}
\definecolor{mauve}{rgb}{0.58,0,0.82}
\lstset{ %
  basicstyle=\footnotesize\monaco,       % the size of the fonts that are used for the code
  numbers=left,                   % where to put the line-numbers
  numberstyle=\footnotesize\monaco\color{gray},  % the style that is used for the line-numbers
  numbersep=5pt
  stepnumber=1,                   % the step between two line-numbers. If it's 1, each line
                                  % will be numbered
  numbersep=5pt,                  % how far the line-numbers are from the code
  backgroundcolor=\color{white},      % choose the background color. You must add \usepackage{color}
  showspaces=false,               % show spaces adding particular underscores
  showstringspaces=false,         % underline spaces within strings
  showtabs=false,                 % show tabs within strings adding particular underscores
  frame=lines,                   % adds a frame around the code
  rulecolor=\color{black},        % if not set, the frame-color may be changed on line-breaks within not-black text (e.g. commens (green here))
  tabsize=4,                      % sets default tabsize to 2 spaces
  captionpos=t,                   % sets the caption-position to bottom
  breaklines=true,                % sets automatic line breaking
  breakatwhitespace=false,        % sets if automatic breaks should only happen at whitespace
  title=\lstname,                   % show the filename of files included with \lstinputlisting;
                                  % also try caption instead of title
  keywordstyle=\color{blue},          % keyword style
  commentstyle=\color{dkgreen},       % comment style
  stringstyle=\color{mauve},         % string literal style
  escapeinside={\%*}{*)},            % if you want to add LaTeX within your code
  morekeywords={*,...}               % if you want to add more keywords to the set
}

\usepackage{amssymb} 
\usepackage{amsmath}
\usepackage[ruled,vlined]{algorithm2e}

\setlength{\parindent}{2em}

\renewcommand{\G}{\mathbb{G}}
\newcommand{\Z}{\mathbb{Z}}
\newcommand{\Q}{\mathbb{Q}}
\newcommand{\F}{\mathbb{F}}

\newcommand{\Sbox}{\textsf{Sbox}}
\newcommand{\code}[1]{\lstinline!#1!}

\newcommand{\CKDpriv}{\textsf{CKDpriv}}
\newcommand{\CKDpub}{\textsf{CKDpub}}

%%%%%%%处理下划线:_%%%%%%%%%
\usepackage{underscore}
%%%%%%%处理下划线:_%%%%%%%%%

\setlength{\parindent}{2.1em}

%%%设置页眉和页码格式
\usepackage{fancyhdr}
\newcommand{\makeheadrule}{%
\rule[0.85\baselineskip]{\headwidth}{0.5pt}\vskip-.8\baselineskip}%1.5 0.4->0.5
\makeatletter
\renewcommand{\headrule}{%
{\if@fancyplain\let\headrulewidth\plainheadrulewidth\fi
\makeheadrule}}
\makeatother
\pagestyle{fancy}
\fancyhf{}
\fancyhead[r]{\textit{Crypto In Action}}
\fancyfoot[C]{--{~\thepage~}--}
%%%设置页眉和页码格式结束

\usepackage{color}
\newcommand{\red}{\textcolor{red}}
\newcommand{\blue}{\textcolor{blue}}

\begin{document}

\title{Monero隐患与ristretto255点群}
\author{longcpp \\ \small{longcpp9@gmail.com}}

\maketitle

\section{余因子为8的现实挑战与对策}

蒙哥马利曲线Curve25519以及扭曲爱德华曲线Edwards25519因为其速度以及易于安全实现等特点,
在区块链领域内外逐渐被广泛采用, 例如基于Curve25519的Diffie-Hellman密钥交换协议X25519被
TLS 1.3协议采纳, Monero中则使用了基于Edwards25519的Schnorr签名机制, 而Tendermint Core
项目的共识投票过程则采用了基于Edwards25519的EdDSA签名机制Ed25519. 
对比基于secp256k1/secp256r1等NIST推荐曲线的ECDH协议/ECDSA/Schnorr等签名机制, 
基于Curve25519/Edwards25519的密钥交换协议或者签名机制在速度安全等方面都胜出.
也因此很多工程项目倾向于基于Curve25519/Edwards25519构建新的密码学协议,
然而由于Edwards25519曲线较少被提及的参数余因子不为1的事实, 导致了CryptoNote协议中基于
该曲线构建RingCT交易时引入了安全漏洞,使得双花甚至多花成为可能,影响了所有基于CryptoNote
协议的数字货币, 例如Monero, Bytecoin等项目. 值得庆幸的是,在该漏洞被利用之前Monero团队就
填补了这个安全漏洞. 对比之下Bytecoin项目就没有这么幸运了, 有攻击者利用这一漏洞构建双花交易,
凭空创建了更多的数字货币.

广泛应用的secp256r1/secp256k1曲线的余因子为1, 天然规避了与余因子相关联的安全隐患,
因此基于这两条曲线的ECDH或者签名机制通常无需考虑余因子的影响.
由于Curve25519/Edwards25519的余因子为8,在设计密码协议时必须将余因子不为1这个事实纳入考量,
例如X25519协议中解码点的倍乘中用到的标量参数时总是会将该标量的最低3比特清零,
而Ed25519签名机制中,从种子(Seed)派生出私钥后也会将私钥的最低3比特清零之后再参与后续计算.
椭圆曲线点群的余因子不为1的干扰可以参考X25519/Ed25519协议的设计,在上层协议的设计中将
底层点群的这一因素纳入考量,并采取相应的措施.然而对于复杂的密码协议,这种方式是一个巨大的安全
挑战, 这点可CryptoNote中发现的相关安全漏洞.在上层协议设计中,不断加入为底层点群的数学结构
采取防范措施,也会使得协议的安全性难以论述.另外在复杂的密码学协议设计中考虑应对措施本身
也是一个技术挑战,参考零知识证明系统Bulletproof. 除了安全性方面的影响之外, 由于Ed25519的私钥
空间不是连续的整数值, 也导致Ed25519签名机制适配分层钱包的机制时遇到了一些技术困难. 

是否存在方法能够既享用Curve25519/Edwards25519的速度优势,又能够解放上层密码协议的设计?
答案是肯定的. Mike Hamberg提出的Decaf方法能够利用商群的概念在余因子为4的非素数阶的点群上
构建素数阶的点群.基于新的素数阶的点群,与余因子不为1相关的安全隐患与技术障碍都得以规避,
并且这一技术并没有引入新的安全假设, 对既有的Curve25519/Edwards25519代码实现的改动也很少.
然而Decaf技术无法直接应用于Curve25519/Edwards25519,因为此处的余因子为8.
Isis Agora Lovecruft和Henry de Valence提出Ristretto技术通过扩展Decaf技术,
同样利用商群的概念可以将余因子为8的非素数阶点群抽象成为素数阶的点群, 
由此上层协议可以安心利用曲线的速度优势且不再被余因子非1的事实所羁绊.
Polkadot项目在其路线图中规划了基于Ristretto的签名机制和可验证随机函数(Verifiable Random 
Function)开发与集成, 而由Interstellar赞助由
Henry de Valence, Cathie Yun和Oleg Andreev所完成的基于Ristretto技术所实现的Bulletproof
(Rust语言实现)是目前效率最高的Bulletproof实现,比基于secp256k1的Bulletproof实现快2倍左右.

本次, 我们首先回顾CryptoNote中相关的安全漏洞, Monero中的补救措施以及Bytecoin中的双花;
然后尝试去理解Ristretto技术是如何在非素数阶的点群中抽象出素数阶点群.关于Ed25519与BIP32等
规范的分层钱包机制的适配,留作后续讨论.

\section{Monero隐患与Bytecoin双花}

Monero是目前市值排名最高的提供链上隐私保护特性的数字货币(根据20191015的coinmarketcap.com
数据显示Monero以\$9亿的市值在所有种类的数字货币中排第14位, Dash排18位, Zcash排30位).
UTXO模型的Monero通过组合多项技术达到同时保护交易发起方, 接收方和交易金额. 
利用Diffie-Hellman密钥交换协议为交易的每个输出都创建唯一的一次性地址
(One-Time Address/Stealth Address)可以隐藏交易的接收方; 
利用Pedersen承诺及范围证明(一开始基于Borromean环签名后升级为Bulletproof技术)可隐藏交易金额;
利用可链接的环签名技术可以将交易的发起方
隐藏在一个群组当中.

\section{Ristretto解放上层密码协议设计}

Decaf的意思为脱因咖啡,而Ristretto则是升级版的Espreesso. 作为Decaf技术的扩展, Ristretto技术
同样利用了雅各比四次形式(Jacobi Quartic), 扭曲的爱德华形式(Twisted Edwards),
以及蒙哥马利形式(Montgomery)三种形式的椭圆曲线以及这些形式之间的同源(Isogeny).

\end{document}

