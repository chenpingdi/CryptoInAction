\documentclass{article}
\usepackage[UTF8, heading = false, scheme = plain]{ctex}

\usepackage{geometry}
\geometry{b5paper,left=2cm,right=2cm,top=2cm,bottom=2cm}

\usepackage{color}
\usepackage{amsfonts}
\usepackage{amsmath}

\linespread{1.5}

\usepackage[colorlinks,
            linkcolor=red,
            anchorcolor=blue,
            citecolor=green
            ]{hyperref}

\usepackage{listings}
\usepackage{fontspec}
\usepackage{graphicx}
\usepackage{algorithm,algorithmic}
\newfontfamily\monaco{Monaco}
\definecolor{dkgreen}{rgb}{0,0.6,0}
\definecolor{gray}{rgb}{0.5,0.5,0.5}
\definecolor{mauve}{rgb}{0.58,0,0.82}
\lstset{ %
  basicstyle=\footnotesize\monaco,       % the size of the fonts that are used for the code
  numbers=left,                   % where to put the line-numbers
  numberstyle=\footnotesize\monaco\color{gray},  % the style that is used for the line-numbers
  numbersep=5pt
  stepnumber=1,                   % the step between two line-numbers. If it's 1, each line
                                  % will be numbered
  numbersep=5pt,                  % how far the line-numbers are from the code
  backgroundcolor=\color{white},      % choose the background color. You must add \usepackage{color}
  showspaces=false,               % show spaces adding particular underscores
  showstringspaces=false,         % underline spaces within strings
  showtabs=false,                 % show tabs within strings adding particular underscores
  frame=lines,                   % adds a frame around the code
  rulecolor=\color{black},        % if not set, the frame-color may be changed on line-breaks within not-black text (e.g. commens (green here))
  tabsize=4,                      % sets default tabsize to 2 spaces
  captionpos=t,                   % sets the caption-position to bottom
  breaklines=true,                % sets automatic line breaking
  breakatwhitespace=false,        % sets if automatic breaks should only happen at whitespace
  title=\lstname,                   % show the filename of files included with \lstinputlisting;
                                  % also try caption instead of title
  keywordstyle=\color{blue},          % keyword style
  commentstyle=\color{dkgreen},       % comment style
  stringstyle=\color{mauve},         % string literal style
  escapeinside={\%*}{*)},            % if you want to add LaTeX within your code
  morekeywords={*,...}               % if you want to add more keywords to the set
}

\usepackage{amssymb} 

\setlength{\parindent}{2em}

\renewcommand{\G}{\mathbb{G}}
\newcommand{\Z}{\mathbb{Z}}
\newcommand{\Q}{\mathbb{Q}}
\newcommand{\F}{\mathbb{F}}

\newcommand{\Sbox}{\textsf{Sbox}}
\newcommand{\code}[1]{\lstinline!#1!}

\newcommand{\CKDpriv}{\textsf{CKDpriv}}
\newcommand{\CKDpub}{\textsf{CKDpub}}

%%%%%%%处理下划线:_%%%%%%%%%
\usepackage{underscore}
%%%%%%%处理下划线:_%%%%%%%%%

\setlength{\parindent}{2.1em}

%%%设置页眉和页码格式
\usepackage{fancyhdr}
\newcommand{\makeheadrule}{%
\rule[0.85\baselineskip]{\headwidth}{0.5pt}\vskip-.8\baselineskip}%1.5 0.4->0.5
\makeatletter
\renewcommand{\headrule}{%
{\if@fancyplain\let\headrulewidth\plainheadrulewidth\fi
\makeheadrule}}
\makeatother
\pagestyle{fancy}
\fancyhf{}
\fancyhead[r]{\textit{Crypto In Action}}
\fancyfoot[C]{--{~\thepage~}--}
%%%设置页眉和页码格式结束

\usepackage{color}
\newcommand{\red}{\textcolor{red}}
\newcommand{\blue}{\textcolor{blue}}


\begin{document}

\title{Ed25519: EdDSA over Edwards25519}
\author{longcpp \\ \small{longcpp9@gmail.com}}

\maketitle

EdDSA算法(Edwards-curve Digital Signature Algorithm)是Bernstein等人\footnote{
Bernstein, Daniel J., Niels Duif, Tanja Lange, Peter Schwabe, and Bo-Yin Yang. 
High-speed high-security signatures. Journal of Cryptographic Engineering 2, no. 2 (2012): 77-89.
\url{https://ed25519.cr.yp.to/ed25519-20110926.pdf}}
在2012年设计的基于扭曲爱德华曲线(Twisted Edwards Curves)的数字签名算法.
EdDSA签名机制是Schnorr签名机制的一个变种,其设计初衷是在不牺牲安全性的前提下提升签名/验签速度.

Ed25519签名算法是基于哈希函数SHA-512和椭圆曲线Curve25519的EdDSA签名机制,旨在提供128比特的安全强度.

Curve25519是定义在有限域$\F_q, \ q = 2^{255}-19$上的扭曲爱德华曲线
$$-x^2 + y^2 = 1 - \frac{121665}{121666}x^2y^2,$$
曲线上点的个数为$\#E(F_q)=2^c\cdot\ell, $

$$
(x_1, y_1) + (x_2, y_2) = \left( \frac{x_1y_2 + y_1x_2}{1 + dx_1x_2y_1y_2}, \frac{y_1y_2 - x_1x_2}{1-dx_1x_2y_1y_2} \right),
$$
并且单位元为点$(0,1)$.

RFC 8032中总结到, EdDSA签名机制的优势在于
\begin{enumerate}
\item EdDSA在多种计算平台上都能达到较高的性能
\item 签名过程中不需要唯一的随机数,能够避免随机数引发的安全问题
\item 对于侧信道攻击等具有更好的免疫效果
\item 公钥和签名值都较小(Ed25519公钥为32个字节,签名值为64字节)
\item 计算公式是完备(Complete),无需对不相信的点执行点的验证操作
\item EdDSA能抵抗碰撞,底层哈希函数的碰撞不会破坏EdDSA签名机制(PureEdDSA)
\end{enumerate}


Montgomery, Peter L. "Speeding the Pollard and elliptic curve methods of factorization." 
Mathematics of computation 48, no. 177 (1987): 243-264. 
\url{https://www.ams.org/journals/mcom/1987-48-177/S0025-5718-1987-0866113-7/S0025--5718-1987-0866113-7.pdf}

RFC 7748\footnote{
RFC 7748. https://tools.ietf.org/html/rfc7748.
\url{https://tools.ietf.org/html/rfc7748}
}
中定义了椭圆曲线Curve25519和Curve448,以及基于这两条曲线的ECDH协议: X25519和X448.
RFC 8032\footnote{
RFC 8032. Edwards-Curve Digital Signature Algorithm (EdDSA).
\url{https://tools.ietf.org/html/rfc8032}
}
描述了EdDSA (Edwards-curve Digital Signature Algorithm)机制并给出了
基于两个椭圆曲线edwards25519和edwards448的EdDSA算法的具体实例化: 
Ed25519, Ed25519ph, Ed25519ctx, Ed448, Ed448ph.
RFC 8031\footnote{
RFC 8031. Curve25519 and Curve448 for the Internet Key Exchange Protocol Version 2 (IKEv2) Key Agreement. 
\url{https://tools.ietf.org/html/rfc8031}
}
中给出了在IKEv2使用Curve25519和Curve448进行临时密钥交换的规范.
RFC 8080\footnote{
RFC 8080. Edwards-Curve Digital Security Algorithm (EdDSA) for DNSSEC. 
\url{https://tools.ietf.org/html/rfc8080}
}
中给出了在DNSSEC中使用Ed25519和Ed448的规范.
RFC 8410\footnote{
RFC 8410. Algorithm Identifiers for Ed25519, Ed448, X25519, and X448 for Use in the Internet X.509 Public Key Infrastructure.
\url{https://tools.ietf.org/html/rfc8410}
}
中为算法Ed25519, Ed448, X25519, X448定义了用于PKI体系的X.509证书的标识符.
RFC 8420\footnote{
RFC 8420. Using the Edwards-Curve Digital Signature Algorithm (EdDSA) in the Internet Key Exchange Protocol Version 2 (IKEv2).
\url{https://tools.ietf.org/html/rfc8420}
}
中给出了在IKEv2中使用EdDSA时Ed25519和Ed448的ASN.1 Objects.
RFC 8446\footnote{
RFC 8446. The Transport Layer Security (TLS) Protocol Version 1.3.
\url{https://tools.ietf.org/html/rfc8446}
}
中TLS1.3的算法套件中包含了Ed25519和Ed448.
RFC 8463\footnote{
RFC 8463. A New Cryptographic Signature Method for DomainKeys Identified Mail (DKIM).
\url{https://tools.ietf.org/html/rfc8463}
}
为DomainKeys Identified Mail (DKIM) (RFC 6376)添加了新的签名算法Ed25519-SHA256


\end{document}
